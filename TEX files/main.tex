\documentclass[a4paper]{article}
\usepackage{standalone}

\usepackage{color}
\usepackage{url}
\usepackage[T2A]{fontenc} % enable Cyrillic fonts
\usepackage[utf8]{inputenc} % make weird characters work
\usepackage{graphicx}
\usepackage[table,xcdraw]{xcolor}
\graphicspath{ {./Slike/} }
\usepackage{float}
\restylefloat{table}

\usepackage[english,serbian]{babel}
%\usepackage[english,serbianc]{babel} %ukljuciti babel sa ovim opcijama, umesto gornjim, ukoliko se koristi cirilica

\usepackage[unicode]{hyperref}
\hypersetup{colorlinks,citecolor=green,filecolor=green,linkcolor=blue,urlcolor=blue}

\usepackage{listings}
%\usepackage{xurl}
\usepackage{pifont}

%\newtheorem{primer}{Пример}[section] %ćirilični primer
\newtheorem{primer}{Primer}[section]


\definecolor{mygreen}{rgb}{0,0.6,0}
\definecolor{mygray}{rgb}{0.5,0.5,0.5}
\definecolor{mymauve}{rgb}{0.58,0,0.82}

\lstset{ 
  backgroundcolor=\color{white},   % choose the background color; you must add \usepackage{color} or \usepackage{xcolor}; should come as last argument
  basicstyle=\scriptsize\ttfamily,        % the size of the fonts that are used for the code
  breakatwhitespace=false,         % sets if automatic breaks should only happen at whitespace
  breaklines=true,                 % sets automatic line breaking
  captionpos=b,                    % sets the caption-position to bottom
  commentstyle=\color{mygreen},    % comment style
  deletekeywords={...},            % if you want to delete keywords from the given language
  escapeinside={\%*}{*)},          % if you want to add LaTeX within your code
  extendedchars=true,              % lets you use non-ASCII characters; for 8-bits encodings only, does not work with UTF-8
  firstnumber=1,                % start line enumeration with line 1000
  frame=single,	                   % adds a frame around the code
  keepspaces=true,                 % keeps spaces in text, useful for keeping indentation of code (possibly needs columns=flexible)
  keywordstyle=\color{blue},       % keyword style
  language=C++,                 % the language of the code
  morekeywords={*,...},            % if you want to add more keywords to the set
  numbers=left,                    % where to put the line-numbers; possible values are (none, left, right)
  numbersep=5pt,                   % how far the line-numbers are from the code
  numberstyle=\tiny\color{mygray}, % the style that is used for the line-numbers
  rulecolor=\color{black},         % if not set, the frame-color may be changed on line-breaks within not-black text (e.g. comments (green here))
  showspaces=false,                % show spaces everywhere adding particular underscores; it overrides 'showstringspaces'
  showstringspaces=false,          % underline spaces within strings only
  showtabs=false,                  % show tabs within strings adding particular underscores
  stepnumber=1,                    % the step between two line-numbers. If it's 1, each line will be numbered
  stringstyle=\color{mymauve},     % string literal style
  tabsize=2,	                   % sets default tabsize to 2 spaces
  title=\lstname                   % show the filename of files included with \lstinputlisting; also try caption instead of title
}

\addtocontents{toc}{\setcounter{tocdepth}{1}} 

\begin{document}

\documentclass{article}
\begin{document}
\title{Informacioni sistem banke\\ \small{Seminarski rad u okviru kursa\\Informacioni sistemi\\ Matematički fakultet}}
\end{document}

\maketitle
\newpage

\section{Uvod}

Rad predstavlja osnovni informacioni sistem banke. Nastao je kao projekat za kurs "Informacioni sistemi" na
Matematičkom fakultetu u Beogradu.

Rad predstavlja univerzalni informacionis sistem banke. Vodili smo se primerom poslovanja "Eurobanke", ali
smo se potrudili da što više uopštimo sam informacioni sistem. Ideja je da se opiše način poslovanja banaka,
kako one intereaguju sa klijentima i kako one funkcionisšu interno. Kako bismo pružili što relevantnije 
informacije, konstantno smo bili u kontaktu sa bankarskim slušbenikom zaposlenim u Eurobanci, ali smo dodatno samostalno
vršili istraživanja što se tiče usluga ostalih banaka.

\subsection{Korišćeni alati}

Radi vizuelnog objašnjenja slučajeva upotrebe korišćeni su sledeći dijagrami:

\begin{enumerate}
\item Dijagrami klasa
\item BPMN dijagrami
\item Dijagarmi toka podataka (DFD)
\end{enumerate}

Kao softevrska rešenja za izradu navedenih dijagrama korišceni su VisalParadigm CE i Modelio.

\section{Analiza sistema}

Bankarski sektor predstavlja značajan deo ekonomije pošto se sa nužnošću korišćenja bankraskih usluga susreću skoro svi.
Analiziranjem sistema banaka ustanovili smo da većina njih trenutno nudi slične usluge. Kako bi bila ispred konkurencije,
privukla nove i zadržala postojeće klijente, banka treba da obezbedi različite usluge u odnosu na konkurenciju.

Tokom analiziranja poslovanja, ustanovili smo i da banke sto vise treba da se okrenu novim tehnologijama jer korisnici žele 
da im je većina usluga dostupna preko telefona ili računara. U informacionom sistemu banke koji smo napravili, 
obrađene su osnovne osobine sistema  bez kojih smatramo da jedna banka ne bi mogla da funkcioniše.

\subsection{Glavni dijagram slučajeva upotrebe}

Najvažnija karakteristika informacionog sistema banke je cuvanje i proveravanje podataka. Za svakog klijenta (fizičko ili
pravno lice) potrebno je da ima sačuvano podatke iz lične karte kao i dodatne podatke o uslugama koje on koristi.
Detljnijom analizom korišćenih usluga, kao i analiziranjem usluga drugih klijenata, banka može da predvidi i preporuči
klijentu usluge za koje misli da će mu biti korisne, a od kojih banka istovremeno može da profitira. Ovaj slučaj
predstavlja napredinij koncept i nije obrađen u ovom radu.

Sve podatke je prvo potrebno proveriti i obraditi. Od vrste usluge zavisi za koliko vremena će se ovo izvršiti. 
Nakon toga se podaci unose u sistem banke i klijent postaje aktivni član.


\section{Slučajevi upotrebe}
\documentclass{article}
\begin{document}
Otvaranje računa

\end{document}

\documentclass{article}
\begin{document}
Transakcije

\end{document}

\documentclass{article}
\begin{document}
Proizvodi

\end{document}


\end{document}
